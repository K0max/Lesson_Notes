\documentclass{article}

\usepackage{NotesTeX}%笔记宏包
\usepackage{ctex}%提供中文
%------------------Math Package-------------------
\usepackage{amsmath}%提供数学等
\usepackage{amssymb}%一些特殊但是也常用的数学符号
\usepackage{amsthm}%数学环境设置
\usepackage{upgreek}%直立体pi等
\usepackage{esvect}%提供向量箭头,命令为\vv{*}
\usepackage{esint}%提供\oiint
\usepackage{extarrows}%提供箭头和等号
\usepackage{bm}%提供数学环境下的粗体
\usepackage{xcolor}
%------------------utility-------------------
%\usepackage[most]{tcolorbox}%提供盒子环境,用来制作theorem等环境
\usepackage{booktabs}%三线表
\usepackage{enumitem}%提供enumerate等环境label的设置
\usepackage{siunitx}%国际制单位
\usepackage{framed}%实现方框效果
%------------------Graphics-------------------
\usepackage{tikz}%TikZ绘图包
\usetikzlibrary{arrows.meta}%提供不同的TikZ箭头
\usetikzlibrary{decorations.pathreplacing}%提供不同的TikZ大括号
%\usepackage{newtxtext}
%------------------Page Design-------------------
\usepackage{geometry}%提供页面设计
\geometry{a4paper,centering,scale=0.8}

\usepackage{lipsum} % 该宏包是通过 \lipsum 命令生成一段本文,即乱序假文,正式使用时不需要引用该宏包

\usepackage{hyperref}%提供超链接

% ------------------Math Theorem environment-------------------
%\newtcbtheorem[number within = section]{definition}{定义}{%
%    boxrule=0pt,
%    boxsep=0pt,
%    colback={white!90},
%    enhanced jigsaw,
%    borderline west={2pt}{0pt}{red},
%    sharp corners,
%    before skip=10pt,
%    after skip=10pt,
%    breakable,
%}{mydefinition}

\newcommand{\const}{\mathrm{const}}
\newcommand{\defeq}{\xlongequal{\mathrm{def}}}%定义符号,也可以直接用amssymb的\triangleq

\newcommand*{\dif}{\mathop{}\!\mathrm{d}}
\DeclareMathOperator{\e}{e}
\renewcommand\vec[1]{\vv{\bm{#1}}}
\newcommand\ceil[1]{\left\lceil#1\right\rceil}
\newcommand\floor[1]{\left\lfloor#1\right\rfloor}
\newcommand\mangle[1]{\langle#1\rangle}
\renewcommand{\oiint}{\varoiint}
\newcommand{\red}{\color{red}}
\newcommand{\blue}{\color{blue}}
\newcommand{\green}{\color{green}}

\renewcommand{\geq}{\geqslant}
\renewcommand{\leq}{\leqslant}
\newcommand*{\E}{\mathbb{E}}
\renewcommand*{\P}{\mathbb{P}}
\newcommand*{\cov}{\mathrm{Cov}}
\renewcommand*{\var}{\mathrm{Var}}
\newcommand*{\iid}{\mathop{}\!\mathrm{i.i.d.}}
\newcommand{\injective}{\hookrightarrow}
\newcommand{\surjective}{\twoheadrightarrow}
\newcommand{\bijective}{\stackrel{\sim}{\rightarrow}}
\everymath{\displaystyle}

\title{\textsc{Signal and System Notes}}
\author{ypa}
\date{2022.12.28\\%
Last Modified:\today}

\newcommand{\Ev}{\mathrm{Ev}}
\newcommand{\Od}{\mathrm{Od}}

\begin{document}
\maketitle
\section{前言}
\noindent 我为什么写这份笔记。众所周知,笔记是写给自己看的,当我学这门课的时候,我发现几个很严重的问题:\\
首先,课本写的非常地厚,不出所料是一堆{\red 屎山}。每个章节有用的东西完全可以浓缩成几页纸几分钟看完。\\
其次,跟上面那条类似,老师讲课的内容也可以在十分钟内看完,但是由于课本的奇妙,很难抓到有用的重点。\\
然后,国内外、国内不同学校{\red 讲课内容不重叠},不同专业之间需要学习的深度不同,不同教材、学校、老师使用的符号系统不同,导致我找同学借来的华科课件失效。\\
做作业的时候发现过于困难,且本门课有期中考,故在学期1/4的时候开始写笔记,旨在把课本的内容浓缩成一份简洁易懂的笔记。目标是{\red 仅需}此份笔记,无需其他一切教材学完这门课。简而言之,就是把这份笔记直接当做{\red 自学教材}而非参考或复习资料\\
那么笔记和课本有啥区别?我觉得电子版最大的优势就是{\red PDF可搜索},可以高效地{\red 面向作业学习}

%\tableofcontents

\section{信号和系统的数学描述及其性质}
\begin{enumerate}[label=(\arabic*)]
  \item 单位阶跃信号:$u(t)=\begin{cases} 1,t>0 \\ 0,t<0 \end{cases}\text{或}\quad
				u[n]=\begin{cases} 1,n{\,\red \geq\,\,}0 \\ 0,n>0 \end{cases}$
	\item 单位冲激信号:离散形式(简单):$\delta[n]=\begin{cases} 1,n=0 \\ 0, n\neq 0 \end{cases}$\\
				连续形式(复杂):Dirac定义:$\begin{cases} \int_{-\infty}^{+\infty}\delta(t)\dif t=1 \\ \delta(t)=0,t\neq 0 \end{cases}$\\
				跟离散形式相比,连续形式的$\delta(0)=\infty$,并且这个无穷是根据积分式定义出来的有意义的无穷
	\item 单位阶跃信号与单位冲激信号的关系:$\delta(t)=\frac{\dif }{\dif t}u(t),\quad u(t)=\int \delta(t)\dif t$
	\item 信号的周期性:$\tilde{x}(t)=\tilde{x}(t\pm T),T\in\mathbb{R}$或$\tilde{x}[n]=\tilde{x}[n\pm N],N\in\mathbb{Z}$
	\item 信号的奇偶性:$x(t)=x_e(t)+x_o(t)$\\
				$x_e(t)=\Ev[x(t)]=\frac{x(t)+x(-t)}{2},x_o(t)=\Od[x(t)]=\frac{x(t)-x(-t)}{2}$
	\item 信号的分类:能量信号和功率信号,分类标准:能量$E_x$和功率$P_x$\\
				$E_x=\int_{-\infty}^{+\infty}|x(t)|^2\dif t\text{或}E_x=\sum_{-\infty}^{\infty}|x[n]|^2$\\
				$P_x=\lim_{T \to \infty}\frac{1}{2T}\int_{-T}^{T}|x(t)|^2\dif t=\lim_{N \to \infty}\frac{1}{2N+1}\sum_{n=-N}^{N}|x[n]|^2$\\
				\begin{enumerate}[label=(\roman*)]
					\item 若$0<E_x<\infty,P\to 0$,则$x(t)$是能量受限信号(简称能量信号)
					\item 若$E_x\to\infty,0<P<\infty$,则$x(t)$是功率受限信号(简称功率信号)
					\item 若$E_x\to\infty,P_x\to\infty$,则称为非能非功信号
				\end{enumerate}
				为啥没有别的情况?用数学计算易证其他情况不存在
	\item 互相关函数与自相关函数:分能量信号和功率信号\\
				\begin{enumerate}
					\item 能量信号:\\
								互相关函数:$R_{xv}(\tau)=\int_{-\infty}^{\infty}x(t+\tau)v^*(t)\dif t \text{或} R_{xv}[m]=\sum_{n=-\infty}^{\infty}x[n+m]v^*[n]$
								自相关函数:$Rx(\tau)=\int_{-\infty}^{+\infty}x(t)x^*(t-\tau)\dif t=\int_{-\infty}^{\infty}x(t)x^*(t-\tau) \text{或} R_x[m]=\sum_{n=-\infty}^{\infty}x[n+m]x^*[n]=\sum_{n=-\infty}^{\infty}x[n]x^*[n-m]$
					\item 功率信号:\\
								互相关函数:$R_{xv}(\tau)=\lim_{T \to \infty}\frac{1}{2T}\int_{-T}^{T}x(t+\tau)v^*(t)\dif t \text{或} R_{xv}[m]=\lim_{N \to \infty}\frac{1}{2N+1}\sum_{n=-T}^{T}x[n+m]v^*[n]$\\
								自相关函数:$Rx(\tau)=\lim_{T \to \infty}\frac{1}{2T}\int_{-T}^{T}x(t)x^*(t-\tau)\dif t=\lim_{N \to \infty}\frac{1}{2N+1}\sum_{n=-T}^{T}x[n+m]x^*[n]$
					\item 周期信号(特殊的功率信号,易证)\\
								$\tilde{R}_x(\tau)=\frac{1}{T}\int_{\mangle{T}}\tilde{x}(t+\tau)\tilde{x}^*(t)\dif t \text{或} \tilde{R}_x[m]=\frac{1}{N}\sum_{n=0}^{N-1}\tilde{x}[n+m]\tilde{x}^*[n]$
				\end{enumerate}
				\begin{remark}
					上述$^*$表示该信号是个复值信号,$^*$即为共轭,如果是实信号,那共轭等于本身;$\mangle{T}$表示任意长度为$T$的周期区间
				\end{remark}
	\item 系统的性质:
				\begin{enumerate}[label=(\roman*)]
					\item 线性性:对$y(t)=x(t)$,令$x(t)=\alpha x_1(t)+\beta x_2(t)$,有$y(t)=\alpha y_1(t)+\beta y_2(t)$\\
								{\red 此外还要求}:零输入零输出。即$y(t)=x(t)+1$是非线性的,因为令$x(t)=0,y(t)=1\neq 0$,所以与其说是线性函数,更像是正比例函数。课本描述如下:\\
								系统$y=T\{x(t)\}$对两个输入$x_1(t)\to y_1(t),x_2(t)\to y_2(t)$,如果同时满足:
								\begin{enumerate}[label=(\alph*)]
									\item 可加性:$x_1(t)+x_2(t)\to y_1(t)+y_2(t)$
									\item 比例性或其次性:$cx_1(t)\to cy_1(t)$
								\end{enumerate}
					\item 记忆性:每一时刻的输出仅取决于同一时刻的输入信号,则称为{\red 无}记忆性(即时系统);否则称为有记忆性(动态系统)\\
								{\red 注意!无记忆性是仅取决于当前,记忆性是取决于当前过去和未来}。课本描述如下:\\
								无记忆系统的输入输出特性:$y=f(x)$
					\item 因果性:任意时刻的输出信号仅取决于当前和过去的输入,与未来的输入无关,则具有因果性(因果系统);否则为非因果。课本描述为\\
								对任何$t$有$y(t)=f\{x(t-\tau),\tau \geq 0\}$
					\item 稳定性:输入信号有界时,输出也有界,则称为稳定;否则称为不稳定
					\item 可逆性:根据系统的输出信号可以唯一确定其输入信号,则称为可逆;否则称为不可逆。简单点说就是一一映射
					\item 时不变性:若输入信号有一个时移,在输出信号有相同的时移,则称为时不变;否则称为时变。课本描述如下:\\
								对$x(t)\to y(t)$,在任何输入信号下,对任何$t_0$都分别有$x(t-t_0)\to y(t-t_0)$,则该系统是时不变的。
				\end{enumerate}
				\begin{remark}
					所有无记忆系统(仅与当前有关)必为因果(与未来无关),但反之未必成立。上述性质理解起来要用信号与系统的思维,输入输出是两串数列,数列的下标即为时间。上述描述时为了省事,仅写出连续信号的情况,离散同理
				\end{remark}
\end{enumerate}

\section{LTI系统的时域分析和信号卷积}
\begin{enumerate}[label=(\arabic*)]
	\item LTI系统意为线性时不变系统(Linear Time-Invariant)
	\item 离散信号可以分解为时移单位冲击的线性组合:$x[n]=\sum_{k=-\infty}^{\infty}x[k]\delta[n-k]$(展开易证),单位阶跃信号展开式:$u[n]=\sum_{k=0}^{\infty}\delta[n-k]$
	\item 卷积运算:$x[n]*v[n]=\sum_{k=-\infty}^{\infty}x[k]v[n-k],\quad x(t)*v(t)=\int_{-\infty}^{+\infty}x(\tau)v(t-\tau)\dif \tau$
	\item 单位冲激响应$h[n]$:当$x[n]=\delta[n]$的时候的输出$h[n]$,即对系统$y[n]=T\{x[n]\}$,有$h[n]=T\{\delta[n]\}$。得到了$h[n]$,则有重要的下式\\
				$\red y[n]=x[n]*h[n]=\sum_{k=-\infty}^{\infty}x[k]h[n-k]$(展开易证)
	\item 卷积运算的性质:
				\begin{enumerate}[label=(\roman*)]
					\item 可交换:$x(t)*h(t)=h(t)*x(t)$
					\item 结合律:$[x(t)*h_1(t)]*h_2(t)=x(t)*[h_1(t)*h_2(t)]$
					\item 分配律:$x(t)*[h_1(t)+h_2(t)]=x(t)*h_1(t)+x(t)*h_2(t)$
					\item 求导:$[x(t)*h(t)]'=x'(t)*h(t)=x(t)*h'(t)$利用这个性质,有推论$x(t)*h(t)=x'(t)*\int h(t)\dif t$\\
								$\frac{\dif^k}{\dif t^k}[x(t)*h(t)]=x(t)*\frac{\dif^k}{\dif t^k}h(t)=\frac{\dif^k}{\dif t^k}x(t)*h(t)$
					\item 积分:$\int_{-\infty}^{t}[x(\tau)*h(\tau)]\dif\tau=x(t)*\left[\int_{-\infty}^{t}h(\tau)\dif\tau\right]=\left[\int_{-\infty}^{t}x(\tau)\dif\tau\right]*h(t)$
					\item 冲激函数的卷积:$x(t)*\delta(t)=x(t)$(别卷了,卷了也是白卷)\\
								延时:$x(t)*\delta(t-t_0)=x(t-t_0)$
					\item 单位冲激响应的延时:对$y(t)=x(t)*h(t)$有$x(t-t_0)*h(t)=x(t)*h(t-t_0)=y(t-t_0)$,此处$h(t)$是单位冲激响应
				\end{enumerate}
				\begin{remark}
					上述未标明是单位冲激响应的$h(t)$则是任意信号,写成$v(t),a(t)$也一样。
				\end{remark}
	\item 做题中常用到的性质:
				\begin{enumerate}[label=(\roman*)]
					\item $u[n]-2u[n-1]+3u[n-2]=\delta[n]-\delta[n-1]+2\delta[n-2]$(因为$n=0,1,2$的时候才不为零,可以列举得到)
					\item $x*h=x'*\int h\dif t$
				\end{enumerate}
	\item 相关函数:在第二章定义过,能量信号的互相关函数为:\\
				$R_{xg}(t) = \int_{-\infty}^{+\infty}x(\tau)g^*(\tau-t)\dif\tau = x(t)*g^*(-t)$
	\item 周期卷积:$\tilde{y}(t)=\tilde{x}_1(t)\circledast\tilde{x}_2(t)=\int_{\mangle{T}}\tilde{x}_1(\tau)\tilde{x}_2(t-\tau)\dif\tau$或\\
				$\tilde{y}[n]=\tilde{x}_1[n]\circledast\tilde{x}_2[n]=\sum_{k\in\mangle{N}}\tilde{x}_1[k]\tilde{x}_2[n-k]$
	\item 单位冲激响应表征的性质:若输出位输入与某个时间函数或序列的卷积,则为LTI。$y(t)=T\{x(t)\}=x(t)*f(t)$,则该系统为LTI,且$h(t)=f(t)$
	\item LTI系统的一些性质:
				\begin{enumerate}[label=(\roman*)]
					\item 记忆性:无记忆性的LTI必须满足:$h(t)=0,(t\neq 0)$
					\item 因果性:LTI的因果性判据:$h(t)=0,(t<0)$
					\item 稳定性:充要条件:$\int_{-\infty}^{\infty}|h(t)|\dif t<\infty$
				\end{enumerate}
\end{enumerate}

\section{用微分方程或差分方程描述的系统}
\subsection{微分方程的解}
\begin{enumerate}[label=(\arabic*)]
	\item 齐次解:$\sum_{i=0}^{n}a_iy^{(i)}(t) = 0$的解(即右边为零的解)\\
				齐次解只由特征根决定:特征方程:$\sum_{i=0}^{n}a_i\lambda^i = 0$即$\lambda^n+a_{n-1}\lambda^{n-1}+\cdots +a_1\lambda+a_0 = 0$
				\begin{enumerate}[label=(\alph*)]
					\item 若特征根均为单根:$y_{H}(t) = \sum_{i=1}^{n}C_i \e^{-\lambda_i t}$
					\item 若存在重根$\lambda_1$,则该$r$重根对应的解为$(C_1t^{r-1}+C_2t^{r-2}+\cdots +C_{r-1}t+C_r)\e^{-\lambda_1 t}$
					\item 若特征根含共轭复根:$\lambda_{1,2} = \alpha \pm j\beta$,则对应的解为:$C_1\e^{-\alpha t}\cos(\beta t)+C_2\e^{-\alpha t}\sin(\beta t)$或$A\e^{-\alpha t}\cos(\beta t-\theta),A\e^{j\theta}=C_1+jC_2$
				\end{enumerate}
				上述$y_{H}$的角标$H$(Homogeneous)代表齐次解,接下来的$y_{P}$的角标$P$(Particular)代表特解
	\item 特解:跟激励的函数形式有关
				\begin{table}[H]
					\centering
					\renewcommand\arraystretch{1.5}
					\begin{tabular}{c|c}
						\toprule
						激励(输入)信号 & 特解$y_{P}(t)$ \\
						\midrule
						$E(\const)$ & $P(\const)$ \\
						$\e^{\alpha t},\red \alpha\neq\lambda_{i}$ & $P(\e^{\alpha t})$ \\
						$\e^{\alpha t},\alpha=\lambda_{i}$,且$\lambda_i$为$\sigma_i$重根 & $\sum_{m=0}^{\sigma_i}P_{m}t^{m}\e^{\alpha t}$ \\
						$\sum_{k=0}^{L}E_{k}t^{k}\e^{\alpha t},\red \alpha\neq\lambda_i$ & $\sum_{m=0}^{L}P_{m}^{m}\e^{\alpha t}$ \\
						$\sum_{k=0}^{L}E_{k}t^{k}\e^{\alpha t},\alpha=\lambda_i$,且$\lambda_i$为$\sigma_i$重根 & $\sum_{m=0}^{L}P_{m}t^{m}\e^{\alpha t}$ \\
						\bottomrule
					\end{tabular}
				\end{table}
	\item 全解:$y(t) = \mathop{y_{H}(t)}\limits_{\substack{\downarrow\\ \text{齐次解} \\ \text{自由响应}}} + \mathop{y_{P}(t)}\limits_{\substack{\downarrow\\ \text{特解} \\ \text{强迫响应}}}$
\end{enumerate}

\appendix
\section{怪话表}
\begin{table}[H]
	\centering
	\begin{tabular}{ccc}
		\toprule
		中文 & 符号 & 说明 \\
		\midrule
		单位冲激函数 & $\delta(t)$ &  \\
		单位阶跃函数 & $u(t)$ & 也有记号$\theta(t),\varepsilon(t)$ \\
		\bottomrule
	\end{tabular}
\end{table}

\end{document}
\documentclass{article}

\usepackage{NotesTeX}%笔记宏包
\usepackage{ctex}%提供中文
%------------------Math Package-------------------
\usepackage{amsmath}%提供数学等
\usepackage{amssymb}%一些特殊但是也常用的数学符号
\usepackage{amsthm}%数学环境设置
\usepackage{upgreek}%直立体pi等
\usepackage{esvect}%提供向量箭头,命令为\vv{*}
\usepackage{esint}%提供\oiint
\usepackage{extarrows}%提供箭头和等号
\usepackage{bm}%提供数学环境下的粗体
\usepackage{xcolor}
%------------------utility-------------------
%\usepackage[most]{tcolorbox}%提供盒子环境,用来制作theorem等环境
\usepackage{booktabs}%三线表
\usepackage{enumitem}%提供enumerate等环境label的设置
\usepackage{siunitx}%国际制单位
\usepackage{framed}%实现方框效果
%------------------Graphics-------------------
\usepackage{tikz}%TikZ绘图包
\usetikzlibrary{arrows.meta}%提供不同的TikZ箭头
\usetikzlibrary{decorations.pathreplacing}%提供不同的TikZ大括号
%\usepackage{newtxtext}
%------------------Page Design-------------------
\usepackage{geometry}%提供页面设计
\geometry{a4paper,centering,scale=0.8}

\usepackage{caption}%可以在minipage中使用\caption命令
%\usepackage{lipsum} % 该宏包是通过 \lipsum 命令生成一段本文,即乱序假文,正式使用时不需要引用该宏包

\usepackage{hyperref}%提供超链接

% ------------------Math Theorem environment-------------------
%\newtcbtheorem[number within = section]{definition}{定义}{%
%    boxrule=0pt,
%    boxsep=0pt,
%    colback={white!90},
%    enhanced jigsaw,
%    borderline west={2pt}{0pt}{red},
%    sharp corners,
%    before skip=10pt,
%    after skip=10pt,
%    breakable,
%}{mydefinition}

\newcommand{\const}{\mathrm{const}}
\newcommand{\defeq}{\xlongequal{\mathrm{def}}}%定义符号,也可以直接用amssymb的\triangleq

\newcommand*{\dif}{\mathop{}\!\mathrm{d}}
\DeclareMathOperator{\e}{e}
\DeclareMathOperator{\bit}{bit}
\DeclareMathOperator{\bits}{bits}
\renewcommand\vec[1]{\vv{\bm{#1}}}
\newcommand\ceil[1]{\left\lceil#1\right\rceil}
\newcommand\floor[1]{\left\lfloor#1\right\rfloor}
\newcommand\mangle[1]{\langle#1\rangle}
\renewcommand{\oiint}{\varoiint}
\newcommand{\red}{\color{red}}
\newcommand{\blue}{\color{blue}}
\newcommand{\green}{\color{green}}

\renewcommand{\geq}{\geqslant}
\renewcommand{\leq}{\leqslant}
\newcommand*{\E}{\mathbb{E}}
\renewcommand*{\P}{\mathbb{P}}
\newcommand*{\cov}{\mathrm{Cov}}
\renewcommand*{\var}{\mathrm{Var}}
\newcommand*{\iid}{\mathop{}\!\mathrm{i.i.d.}}
\newcommand{\injective}{\hookrightarrow}
\newcommand{\surjective}{\twoheadrightarrow}
\newcommand{\bijective}{\stackrel{\sim}{\rightarrow}}
\setlength\parindent{0pt}
\everymath{\displaystyle}

\title{\textsc{Signal and System Notes}}
\author{ypa}
\date{2022.12.28\\%
Last Modified:\today}

\newcommand\Ev[1]{\mathrm{Ev}\left[#1\right]}
\newcommand\Od[1]{\mathrm{Od}\left[#1\right]}
\newcommand\Sa[1]{\mathrm{Sa}\left(#1\right)}
\newcommand\zi{\mathrm{zi}}
\newcommand\zs{\mathrm{zs}}
\renewcommand\var[1]{\Delta{#1}}
\newcommand\inv{\mathrm{inv}}

\begin{document}
\maketitle
\section{前言}
我为什么写这份笔记。众所周知,笔记是写给自己看的,当我学这门课的时候,我发现几个很严重的问题:\\
首先,课本写的非常地厚,不出所料是一堆{\red 屎山}。每个章节有用的东西完全可以浓缩成几页纸几分钟看完。\\
其次,跟上面那条类似,老师讲课的内容也可以在十分钟内看完,但是由于课本的奇妙,很难抓到有用的重点。\\
然后,国内外、国内不同学校{\red 讲课内容不重叠},不同专业之间需要学习的深度不同,不同教材、学校、老师使用的符号系统不同,导致我找同学借来的华科课件失效。\\
做作业的时候发现过于困难,且本门课有期中考,故在学期1/4的时候开始写笔记,旨在把课本的内容浓缩成一份简洁易懂的笔记。目标是{\red 仅需}此份笔记,无需其他一切教材学完这门课。简而言之,就是把这份笔记直接当做{\red 自学教材}而非参考或复习资料\\
那么笔记和课本有啥区别?我觉得电子版最大的优势就是{\red PDF可搜索},可以高效地{\red 面向作业学习}\\
最后的怪话表说白了就是符号表,涵盖本书的符号体系和其他教材的符号体系,防止各位听外校网课时一头雾水。

%\tableofcontents

\section{信号和系统的数学描述及其性质}
\begin{enumerate}[label=(\arabic*)]
  \item 单位阶跃信号:$u(t)=\begin{cases} 1,t>0 \\ 0,t<0 \end{cases}\text{或}\quad
				u[n]=\begin{cases} 1,n{\,\red \geq\,\,}0 \\ 0,n>0 \end{cases}$
	\item 单位冲激信号:离散形式(简单):$\delta[n]=\begin{cases} 1,n=0 \\ 0, n\neq 0 \end{cases}$\\
				连续形式(复杂):Dirac定义:$\begin{cases} \int_{-\infty}^{+\infty}\delta(t)\dif t=1 \\ \delta(t)=0,t\neq 0 \end{cases}$\\
				跟离散形式相比,连续形式的$\delta(0)=\infty$,并且这个无穷是根据积分式定义出来的有意义的无穷
	\item 单位阶跃信号与单位冲激信号的关系:$\delta(t)=\frac{\dif }{\dif t}u(t),\quad u(t)=\int \delta(t)\dif t$
	\item 信号的周期性:$\tilde{x}(t)=\tilde{x}(t\pm T),T\in\mathbb{R}$或$\tilde{x}[n]=\tilde{x}[n\pm N],N\in\mathbb{Z}$
	\item 信号的奇偶性:$x(t)=x_e(t)+x_o(t)$\\
				$x_e(t)=\Ev{x(t)}=\frac{x(t)+x(-t)}{2},x_o(t)=\Od{x(t)}=\frac{x(t)-x(-t)}{2}$
	\item 信号的分类:能量信号和功率信号,分类标准:能量$E_x$和功率$P_x$\\
				$E_x=\int_{-\infty}^{+\infty}|x(t)|^2\dif t\text{或}E_x=\sum_{-\infty}^{\infty}|x[n]|^2$\\
				$P_x=\lim_{T \to \infty}\frac{1}{2T}\int_{-T}^{T}|x(t)|^2\dif t=\lim_{N \to \infty}\frac{1}{2N+1}\sum_{n=-N}^{N}|x[n]|^2$\\
				\begin{enumerate}[label=(\roman*)]
					\item 若$0<E_x<\infty,P\to 0$,则$x(t)$是能量受限信号(简称能量信号)
					\item 若$E_x\to\infty,0<P<\infty$,则$x(t)$是功率受限信号(简称功率信号)
					\item 若$E_x\to\infty,P_x\to\infty$,则称为非能非功信号
				\end{enumerate}
				为啥没有别的情况?用数学计算易证其他情况不存在
	\item 互相关函数与自相关函数:分能量信号和功率信号
				\begin{enumerate}
					\item 能量信号:\\
								互相关函数:$R_{xv}(\tau)=\int_{-\infty}^{\infty}x(t+\tau)v^*(t)\dif t \text{或} R_{xv}[m]=\sum_{n=-\infty}^{\infty}x[n+m]v^*[n]$
								自相关函数:$Rx(\tau)=\int_{-\infty}^{+\infty}x(t)x^*(t-\tau)\dif t=\int_{-\infty}^{\infty}x(t)x^*(t-\tau) \text{或} R_x[m]=\sum_{n=-\infty}^{\infty}x[n+m]x^*[n]=\sum_{n=-\infty}^{\infty}x[n]x^*[n-m]$
					\item 功率信号:\\
								互相关函数:$R_{xv}(\tau)=\lim_{T \to \infty}\frac{1}{2T}\int_{-T}^{T}x(t+\tau)v^*(t)\dif t \text{或} R_{xv}[m]=\lim_{N \to \infty}\frac{1}{2N+1}\sum_{n=-T}^{T}x[n+m]v^*[n]$\\
								自相关函数:$Rx(\tau)=\lim_{T \to \infty}\frac{1}{2T}\int_{-T}^{T}x(t)x^*(t-\tau)\dif t=\lim_{N \to \infty}\frac{1}{2N+1}\sum_{n=-T}^{T}x[n+m]x^*[n]$
					\item 周期信号(特殊的功率信号,易证)\\
								$\tilde{R}_x(\tau)=\frac{1}{T}\int_{\mangle{T}}\tilde{x}(t+\tau)\tilde{x}^*(t)\dif t \text{或} \tilde{R}_x[m]=\frac{1}{N}\sum_{n=0}^{N-1}\tilde{x}[n+m]\tilde{x}^*[n]$
				\end{enumerate}
				\begin{remark}
					上述$^*$表示该信号是个复值信号,$^*$即为共轭,如果是实信号,那共轭等于本身;$\mangle{T}$表示任意长度为$T$的周期区间
				\end{remark}
	\item 系统的性质:
				\begin{enumerate}[label=(\roman*)]
					\item 线性性:对$y(t)=x(t)$,令$x(t)=\alpha x_1(t)+\beta x_2(t)$,有$y(t)=\alpha y_1(t)+\beta y_2(t)$\\
								{\red 此外还要求}:零输入零输出。即$y(t)=x(t)+1$是非线性的,因为令$x(t)=0,y(t)=1\neq 0$,所以与其说是线性函数,更像是正比例函数。课本描述如下:\\
								系统$y=T\{x(t)\}$对两个输入$x_1(t)\to y_1(t),x_2(t)\to y_2(t)$,如果同时满足:
								\begin{enumerate}[label=(\alph*)]
									\item 可加性:$x_1(t)+x_2(t)\to y_1(t)+y_2(t)$
									\item 比例性或其次性:$cx_1(t)\to cy_1(t)$
								\end{enumerate}
					\item 记忆性:每一时刻的输出仅取决于同一时刻的输入信号,则称为{\red 无}记忆性(即时系统);否则称为有记忆性(动态系统)\\
								{\red 注意!无记忆性是仅取决于当前,记忆性是取决于当前过去和未来}。课本描述如下:\\
								无记忆系统的输入输出特性:$y=f(x)$
					\item 因果性:任意时刻的输出信号仅取决于当前和过去的输入,与未来的输入无关,则具有因果性(因果系统);否则为非因果。课本描述为\\
								对任何$t$有$y(t)=f\{x(t-\tau),\tau \geq 0\}$
					\item 稳定性:输入信号有界时,输出也有界,则称为稳定;否则称为不稳定
					\item 可逆性:根据系统的输出信号可以唯一确定其输入信号,则称为可逆;否则称为不可逆。简单点说就是一一映射
					\item 时不变性:若输入信号有一个时移,在输出信号有相同的时移,则称为时不变;否则称为时变。课本描述如下:\\
								对$x(t)\to y(t)$,在任何输入信号下,对任何$t_0$都分别有$x(t-t_0)\to y(t-t_0)$,则该系统是时不变的。
								\begin{example}
									$y[n] = (\frac{1}{2})^n\sum_{k=-\infty}^{n}2^kx[k]$\\
									先时移后输入:$x[n]\xrightarrow{n-n_0}x[n-n_0]\longrightarrow (\frac{1}{2})^n\sum_{k=-\infty}^{n}2^kx[k-n_0]\xrightarrow{i=k-n_0}$\\
									$(\frac{1}{2})^n\sum_{i=-\infty}^{n-n_0}2^{i+n_0}x[i] = (\frac{1}{2})^{n-n_0}\sum_{i=-\infty}^{n-n_0}2^ix[i]$\\
									先输入后时移:$x[n]\longrightarrow (\frac{1}{2})\sum_{k=-\infty}^{n}2^kx[k]\xrightarrow{n-n_0}(\frac{1}{2})^{n-n_0}\sum_{k=-\infty}^{n-n_0}2^kx[k]$
								\end{example}
				\end{enumerate}
				\begin{remark}
					所有无记忆系统(仅与当前有关)必为因果(与未来无关),但反之未必成立。上述性质理解起来要用信号与系统的思维,输入输出是两串数列,数列的下标即为时间。上述描述时为了省事,仅写出连续信号的情况,离散同理
				\end{remark}
\end{enumerate}

\section{LTI系统的时域分析和信号卷积}
\begin{enumerate}[label=(\arabic*)]
	\item LTI系统意为线性时不变系统(Linear Time-Invariant)
	\item 离散信号可以分解为时移单位冲击的线性组合:$x[n]=\sum_{k=-\infty}^{\infty}x[k]\delta[n-k]$(展开易证),单位阶跃信号展开式:$u[n]=\sum_{k=0}^{\infty}\delta[n-k]$
	\item 卷积运算:$x[n]*v[n]=\sum_{k=-\infty}^{\infty}x[k]v[n-k],\quad x(t)*v(t)=\int_{-\infty}^{+\infty}x(\tau)v(t-\tau)\dif \tau$
	\item 单位冲激响应$h[n]$:当$x[n]=\delta[n]$的时候的输出$h[n]$,即对系统$y[n]=T\{x[n]\}$,有$h[n]=T\{\delta[n]\}$。得到了$h[n]$,则有重要的下式\\
				$\red y[n]=x[n]*h[n]=\sum_{k=-\infty}^{\infty}x[k]h[n-k]$(展开易证)
	\item 卷积运算的性质:
				\begin{enumerate}[label=(\roman*)]
					\item 可交换:$x(t)*h(t)=h(t)*x(t)$
					\item 结合律:$[x(t)*h_1(t)]*h_2(t)=x(t)*[h_1(t)*h_2(t)]$
					\item 分配律:$x(t)*[h_1(t)+h_2(t)]=x(t)*h_1(t)+x(t)*h_2(t)$
					\item 求导:$[x(t)*h(t)]'=x'(t)*h(t)=x(t)*h'(t)$利用这个性质,有推论$x(t)*h(t)=x'(t)*\int h(t)\dif t$\\
								$\frac{\dif^k}{\dif t^k}[x(t)*h(t)]=x(t)*\frac{\dif^k}{\dif t^k}h(t)=\frac{\dif^k}{\dif t^k}x(t)*h(t)$
					\item 积分:$\int_{-\infty}^{t}[x(\tau)*h(\tau)]\dif\tau=x(t)*\left[\int_{-\infty}^{t}h(\tau)\dif\tau\right]=\left[\int_{-\infty}^{t}x(\tau)\dif\tau\right]*h(t)$
					\item 冲激函数的卷积:$x(t)*\delta(t)=x(t)$({\red 别卷了,卷了也是白卷})\\
								延时:$x(t)*\delta(t-t_0)=x(t-t_0)$\\
								导数:$\delta'(t)*x(t) = [\delta(t)*x(t)]' = x'(t)$
					\item 单位冲激响应的延时:对$y(t)=x(t)*h(t)$有$x(t-t_0)*h(t)=x(t)*h(t-t_0)=y(t-t_0)$,此处$h(t)$是单位冲激响应
				\end{enumerate}
				\begin{remark}
					上述未标明是单位冲激响应的$h(t)$则是任意信号,写成$v(t),a(t)$也一样。
				\end{remark}
	\item 做题中常用到的性质:
				\begin{enumerate}[label=(\roman*)]
					\item $u[n]-2u[n-1]+3u[n-2]=\delta[n]-\delta[n-1]+2\delta[n-2]$(因为$n=0,1,2$的时候才不为零,可以列举得到)
					\item $x*h=x'*\int h\dif t$
				\end{enumerate}
	\item 基本信号的卷积表:
				\begin{table}[H]
					\begin{tabular}{ccc|ccc}
						\toprule
						\multicolumn{3}{c}{连续时间卷积积分} & \multicolumn{3}{c}{离散时间卷积和} \\
						\midrule
						$x(t)$ & $h(t)$ & $x(t)*h(t)$ & $x[n]$ & $h[n]$ & $x[n]*h[n]$ \\
						$x(t)$ & $\delta(t)$ & $x(t)$ & $x[n]$ & $\delta[n]$ & $x[n]$ \\
						$x(t)$ & $u(t)$ & $\int_{-\infty}^{t}x(\tau)\dif\tau$ & $x[n]$ & $u[n]$ & $\sum_{k=-\infty}^{n}x[k]$ \\
						$x(t)$ & $\delta'(t)$ & $x'(t)$ & $x[n]$ & $\var{\delta[n]}$ & $x[n]-x[n-1]$ \\
						$u(t)$ & $u(t)$ & $tu(t)$ & $u[n]$ & $u[n]$ & $(n+1)u[n]$ \\
						$\e^{-at}u(t)$ & $u(t)$ & $\frac{1-\e^{-at}}{a}u(t)$ & $a^{n}u[n]$ & $u[n]$ & $\frac{1-a^{n+1}}{1-a}u[n]$ \\
						$\sin(\omega t)u(t)$ & $u(t)$ & $\frac{1-\cos (\omega t)}{\omega}u(t)$ & $\sin(\Omega n)u[n]$ & $u[n]$ & ~ \\
						$\cos(\omega t)u(t)$ & $u(t)$ & $\frac{\sin(\omega t)}{\omega}u(t)$ & $\cos(\Omega n)u[n]$ & $u[n]$ & ~ \\
						$\e^{-at}u(t)$ & $\e^{-at}u(t)$ & $t\e^{-at}u(t)$ & $a^{n}u[n]$ & $a^{n}u[n]$ & $(n+1)a^{n}u[n]$ \\
						$\e^{-at}u(t)$ & $\e^{-bt}u(t)$ & $\frac{\e^{-at}-\e^{-bt}}{b-a}u(t)$ & $a^{n}u[n]$ & $b^{n}u[n]$ & $\frac{b^{n+1}-a^{n+1}}{b-a}u[n]$ \\
						\bottomrule
					\end{tabular}
				\end{table}
	\item 相关函数:在第二章定义过,能量信号的互相关函数为:\\
				$R_{xg}(t) = \int_{-\infty}^{+\infty}x(\tau)g^*(\tau-t)\dif\tau = x(t)*g^*(-t)$
	\item 周期卷积:$\tilde{y}(t)=\tilde{x}_1(t)\circledast\tilde{x}_2(t)=\int_{\mangle{T}}\tilde{x}_1(\tau)\tilde{x}_2(t-\tau)\dif\tau$或\\
				$\tilde{y}[n]=\tilde{x}_1[n]\circledast\tilde{x}_2[n]=\sum_{k\in\mangle{N}}\tilde{x}_1[k]\tilde{x}_2[n-k]$
	\item 单位冲激响应表征的性质:若输出位输入与某个时间函数或序列的卷积,则为LTI。$y(t)=T\{x(t)\}=x(t)*f(t)$,则该系统为LTI,且$h(t)=f(t)$
	\item LTI系统的一些性质:
				\begin{enumerate}[label=(\roman*)]
					\item 记忆性:无记忆性的LTI必须满足:$h(t)=0,(t\neq 0)$
					\item 因果性:LTI的因果性判据:$h(t)=0,(t<0)$
					\item 稳定性:充要条件:$\int_{-\infty}^{\infty}|h(t)|\dif t<\infty$
					\item 可逆性:对可逆的LTI系统有:$h_{\inv}(t)*h(t) = \delta(t)$
								\begin{example}
									$y(t) = \e^{-2t}\int_{-\infty}^{t}(\e^{\tau})^2x(\tau)\dif \Rightarrow h(t)=\e^{-2t}\int_{-\infty}^{t}\e^{2\tau}\delta(\tau)\dif\tau = \e^{2t}\cdot\e^{2\cdot 0}\int_{0}^{0}\delta(\tau)\dif t = \e^{-2t}u(t)$\\
									求其逆系统$h_{\inv}(t)*h(t) = \delta(t)$,对$h(t)$求导有$h'(t) = \e^{-2t}\delta(t) -2\e^{-2t}u(t) = \delta(t)-2\e^{-2t}u(t) = \delta(t)-2h(t)\Longrightarrow h'(t)+2h(t) = \delta(t)$\\
									$\Longrightarrow h(t)*(\delta'(t)+2\delta(t)) = \delta(t)\Longrightarrow h_{\inv}(t) = \delta'(t)+2\delta(t)$
								\end{example}
				\end{enumerate}
\end{enumerate}

\section{用微分方程或差分方程描述的系统}
对于一个微(差)分方程,有两种解法。一是数学上的经典解:齐次解$y_{H}$和特解$y_{P}$。二是信号与系统上的零输入响应$y_{\zi}$和零状态响应$y_{\zs}$\\
{\red 杜宝(助教)建议}用零输入和零状态搞,别老惦记你那特解
\begin{remark}
	这两种解法分别有结构:全解=齐次解+特解;完全响应=零输入响应+零状态响应。他们的关系为全解=完全响应,即只是两个解法的不同名字。\\
	齐次解=零输入响应+零状态响应的一部分;特解=零输入响应的一部分\\
	齐次解又叫自由响应或固有响应,特解又叫强迫响应。\\
	更详细的对比请见\href{https://zhuanlan.zhihu.com/p/415103392}{知乎}
\end{remark}
\subsection{齐次解和特解的解法}
\begin{enumerate}[label=(\arabic*)]
	\item 齐次解:$\sum_{i=0}^{N}a_iy^{(i)}(t) = 0$的解(即右边为零的解)\\
			齐次解只由特征根决定:特征方程:$\sum_{i=0}^{n}a_i\lambda^i = 0$即$\lambda^n+a_{n-1}\lambda^{n-1}+\cdots +a_1\lambda+a_0 = 0$
			\begin{enumerate}[label=(\alph*)]
				\item 若特征根均为单根:$y_{H}(t) = \sum_{i=1}^{n}A_i \e^{-\lambda_i t}$
				\item 若存在重根$\lambda_1$,则该$r$重根对应的解为$(A_1t^{r-1}+A_2t^{r-2}+\cdots +A_{r-1}t+A_r)\e^{-\lambda_1 t}$
				\item 若特征根含共轭复根:$\lambda_{1,2} = \alpha \pm j\beta$,则对应的解为:$C_1\e^{-\alpha t}\cos(\beta t)+C_2\e^{-\alpha t}\sin(\beta t)$或$A\e^{-\alpha t}\cos(\beta t-\theta),A\e^{j\theta}=C_1+jC_2$
			\end{enumerate}
			上述$y_{H}$的角标$H$(Homogeneous)代表齐次解,接下来的$y_{P}$的角标$P$(Particular)代表特解\\
			然后再特地提一下差分方程跟微分方程不同的地方:\\
			差分方程$\sum_{k=0}^{n}b_{k}y[n-k] = 0$,则特征根方程为$\sum_{k=0}^{n}b_{k}\lambda^{n-k} = 0$。且最后的齐次解为$y_{H}[n] = \sum_{i=1}^{n}B_i\lambda_i^{n}$。\\
			举例为$y[n]+2[n-1]-3y[n-2] = 0$,特征方程为$\lambda^2+2\lambda-3 = 0\Rightarrow \lambda=-3,1$,$y_{H}[n] = B_i\lambda_i^{n}$。\\
			如果是$y''(t)+2y'(t)-3 = 0$,那也是$\lambda^2+2\lambda-3 = 0$\\
			一般来说,连续时间系统的求导$y'(t)$基本等同于离散时间系统的时延$y[n-1]$,但是在特征根里面刚好不是
	\item 特解:跟激励的函数形式有关
				\begin{table}[H]
					\centering
					\renewcommand\arraystretch{1.5}
					\begin{tabular}{c|c}
						\toprule
						激励(输入)信号 & 特解$y_{P}(t)$ \\
						\midrule
						$E(\const)$ & $P(\const)$ \\
						$\e^{\alpha t},\red \alpha\neq\lambda_{i}$ & $P(\e^{\alpha t})$ \\
						$\e^{\alpha t},\alpha=\lambda_{i}$,且$\lambda_i$为$\sigma_i$重根 & $\sum_{m=0}^{\sigma_i}P_{m}t^{m}\e^{\alpha t}$ \\
						$\sum_{k=0}^{L}E_{k}t^{k}\e^{\alpha t},\red \alpha\neq\lambda_i$ & $\sum_{m=0}^{L}P_{m}^{m}\e^{\alpha t}$ \\
						$\sum_{k=0}^{L}E_{k}t^{k}\e^{\alpha t},\alpha=\lambda_i$,且$\lambda_i$为$\sigma_i$重根 & $\sum_{m=0}^{L}P_{m}t^{m}\e^{\alpha t}$ \\
						\bottomrule
					\end{tabular}
				\end{table}
	\item 全解:$y(t) = \mathop{y_{H}(t)}\limits_{\substack{\downarrow\\ \text{齐次解} \\ \text{自由响应}}} + \mathop{y_{P}(t)}\limits_{\substack{\downarrow\\ \text{特解} \\ \text{强迫响应}}}$
	\item {\red 注:}对差分方程的齐次解系数,课本上标注为$B_{i}$以示和微分方程齐次解的系数$A_{i}$不同。系数$A_{i}$和$B_{i}$都可以为复数
\end{enumerate}

\subsection{零输入响应和零状态响应的解法}
\begin{enumerate}[label=(\arabic*)]
	\item $y(t) = y_{\zi}(t)+y_{\zs}(t)$,其中角标$\zi$(Zero Input)代表零输入响应,角标$\zs$(Zero State)代表零状态响应。为了表示区别,我将zi和zs写成了正体,并且在zi和$i$之间加了逗号
	\item 零输入响应:$y_{\zi}(t) = \sum_{i=1}^{n}A_{\zi,i}\e^{-\lambda_i t}$,和齐次解一样是解等式左边的特征方程。\\
				系数为$A_{\zi,i}$意为第$i$个特征根对应的零输入响应的系数,同样,课本上对差分方程系数表示为$B_{\zi,i}$
				\begin{remark}
					那问题来了,这和齐次解有啥区别。见课本$\mathbf{P}_{122}$的4.5节上方最后一段。摘录一下:虽然自由响应和零输入响应都满足齐次方程的解,但是它们的{\red 系数}却不相同,且代表不同的物理含义。$A_{\zi,i},B_{\zi,i}$仅分别由{\red 起始}条件决定,而$A_{i}$由{\red 起始}条件和输入信号共同决定:在{\red 起始}条件为零的情况下,零输入响应为0,然而自由响应(齐次解)却不等于0,只有输入信号也为0时,自由响应才为0。
				\end{remark}
	\item {\red 起始}条件与{\red 初始}条件的区别:\\
				{\red 起始}条件:$y^{(k)}(0_{\red-}) = c_k\quad (k=0,1,\cdots ,N-1)$\\
				{\red 初始}条件:$y^{(k)}(0^{\red+}) = c_k\quad (k=0,1,\cdots ,N-1)$\\
				诶诶,怎么看不出哪里有区别?哦,原来是0的{\red 角标}不同。由于$y^{(k)}(t)$在$t=0$可能有跃变或者冲激,所以这吊玩意可能不连续,即$y^{(k)}(0^{+})\neq y^{(k)}(0_{-})$\\
				起始条件和初始条件的转换:\\
				对于离散时间系统(差分方程),使用课本$\mathbf{P}_{117}$的4.3.3节的递推算法求得。对连续时间系统(微分方程)就有亿点麻烦了,需要用具体的物理系统的物理原理。零输入和零状态法就是为了{\red 一定程度上规避}起始条件到初始条件的转换提出来的。
	\item 上面已经提到$y(t) = y_{\zi}(t)+y_{\zs}(t) = \sum_{i=1}^{N}A_{\zi,i}\e^{\lambda_i t}+x(t)*h(t)\quad (t\geq 0)$,所以接下来直接求$y_{\zs}(t)$。为了求得零状态响应,我们需要一个类似于Lemma的东西:单位冲激响应
\end{enumerate}
\subsubsection{微分方程表征的LTI系统的单位冲激响应的求法}
\begin{enumerate}[label=(\arabic*)]
	\item 两个因果LTI系统级联的方法:\\
				对原式$\sum_{k=0}^{N}a_k y^{(k)}(t) = \sum_{k=0}^{M}b_kx^{(k)}(t)$,要求单位冲激响应,那直接$x(t)=\delta(t),y(t)=h(t)$,于是有
				\[\sum_{k=0}^{N}a_k h^{(k)}(t) = \sum_{k=0}^{M}b_k\delta^{(k)}(t)\]
				拆分成$h(t)=h_1(t)*h_2(t)$,其中$h_1(t) = \sum_{k=0}^{M}b_{k}\delta ^{(k)}(t)$,即上式右边。
				$h_2(t)$是$\begin{cases}\sum_{k=0}^{N}a_{k}h_2^{(k)}(t) = \delta(t) \\ \text{{\red 起始}条件:}h_2^{(k)}(0_{-}) = 0 \\ k={\red 0},1,\cdots {\red N-1}\end{cases}$的解,且解为:(下式即为{\red 初始}条件)\\
				\inlineequation[微分初始]{h_2^{(N-1)}(0^{+}) = \frac{1}{a_{N}},h_2^{(k)}(0^{+}) \equiv  0,k={\red 0}\cdots {\red N-2}}\\
				对差分方程$\begin{cases}\sum_{k=0}^{N}a_{k}h_2[n-k] = \delta[n] \\ \text{{\red 起始}条件}h_2[-k] = 0 \\ k={\red 1},2,\cdots {\red N}\end{cases}$(下式即为{\red 初始}条件)\\
				解为:\inlineequation[差分初始]{h_2[0]=\frac{1}{a_{0}},h[-k]\equiv 0,k={\red 0},\cdots {\red N-1}}
				\begin{remark}
					注意到上述还是有由起始条件转化为初始条件,但是很简单,初始条件由公式\eqref{微分初始}和\eqref{差分初始}给出
				\end{remark}
				\begin{example}
					求$y''(t)+4y'(t)+3y(t) = x'(t)+2x(t)$的单位冲激响应\\
					\textbf{Solution:\\}
					$h_1(t) = \delta'(t)+2\delta(t)$
					\[\begin{cases}
						h_2''(t)+4h_2'(t)+3h_2(t) = 0 \\
						\text{\red 初始:}h_2(0^{+}) = 0,h_2'(0^{+}) = 1
					\end{cases} \xrightarrow{\text{ODE特征根法}} h_2(t) = \begin{cases}
						A_1\e^{-t}+A_2\e^{-3t}, t>0 \\
						0, t<0
					\end{cases}\]
					再代入初始条件(见\eqref{微分初始})求得$A_1=\frac{1}{2},A_2=-\frac{1}{2}$,则$h_2(t) = \frac{\e^{-t}-\e^{-3t}}{2}$
					于是\[\begin{aligned}
						h(t) &= h_1(t)*h_2(t) = [\delta'(t)+2\delta(t)]*\left\{\frac{\e^{-t}-\e^{-3t}}{2}u(t)\right\} = \frac{\e^{-t}+\e^{-3t}}{2}u(t)\\
						&= [\delta'(t)+2\delta(t)]*h_2(t) = \delta'(t)*h_2(t)+2\delta(t)*h_2(t)\\
						&= (\delta(t)*h_2(t))' + 2h_2(t)\\
						&= h_2'(t) + 2h_2(t)
					\end{aligned}\]
					$y_{\zs} = x(t)*h(t),\quad y(t) = y_{\zi}(t)+y_{\zs}(t) = \sum_{i=0}^{N}A_{\zi,i}\e^{\lambda_i t}s+x(t)*h_1(t)*h_2(t)$
				\end{example}
	\item 还有一种微分和差分方程量变函数项匹配的方法(待定系数法),课本$\mathbf{P}_{125}$,写起来巨麻烦,计算量极大,不写了。
\end{enumerate}
\subsubsection{零状态响应}
现在正式进入零状态响应的求法
\begin{enumerate}[label=(\arabic*)]
	\item $y_{\zs}(t) = x(t)*h(t) = x(t)*h_1(t)*h_2(t)$,好了,这就是零状态
	\item 全响应为$y(t) = y_{\zi}(t)+y_{\zs}(t) = \sum_{i=0}^{N}A_{\zi,i}\e^{\lambda_i t}s+x(t)*h_1(t)*h_2(t)$
\end{enumerate}

\subsection{微分和差分方程表征的因果LTI系统的直接实现结构}
本节名字又臭又长,说人话就是画框图。框图画法在笔记里打真的会去世,所以作罢,老老实实问同学罢,听人讲+自己做几道例题,半小时内就能搞定。

\section{信号和系统的变换域表示法}
\subsection{LTI系统对复指数信号的响应}
若输入为$x(t) = \e^{st}$,易证$y(t) = h(t)*x(t) = \e^{st}\int_{-\infty}^{+\infty}h(\tau)\e^{-s\tau}\dif\tau$\\
系统函数$H(s)$为:$H(s) = \int_{-\infty}^{+\infty}h(\tau)\e^{-s\tau}\dif\tau,s\in\mathbb{C},y(t) = H(s)\e^{st}$\\
对CT(continuous time)周期信号$\tilde{x}(t)$的Fourier级数(CFS):$\tilde{x}(t),\omega_0 .\frac{2\uppi}{T},f_0 = \frac{1}{T}$,有
\[\tilde{x}(t) = \sum_{k=-\infty}^{+\infty}a_{k}\e^{jk\omega_0t},a_{k} = \frac{1}{T}\int_{T}\tilde{x}(t)\e^{-jk\omega_0t}\]
对$\tilde{x}(t)$为实信号的性质:
\begin{enumerate}[label=(\arabic*)]
	\item 若$\tilde{x}(t)$为实信号,有$\tilde{x}(t)=\tilde{x}^{*}(t),a_{k}=a_{-k}^{*}$
	\item \[\begin{aligned}
					\tilde{x}(t) &= \sum_{k=1}^{-\infty}a_k\e^{-jk\omega_0t}+a_0+\sum_{k=1}^{-\infty}s_k\e^{jk\omega_0t} = a_0+\sum_{k=1}^{-\infty}(a_{k}\e^{-jk\omega_0t}+a_{-k}\e^{jk\omega_0t}) \\
					&= a_0+\sum_{k=1}^{+\infty}2\Re{a_{k}\e^{jk\omega_0t}}\\
					\tilde{x}(t) &= a_0+\sum_{k=1}^{+\infty}2|a_{k}|\cos (k\omega_0t+\theta_k)
				\end{aligned}\]
\end{enumerate}
傅里叶级数的敛散性:
\begin{enumerate}[label=(\arabic*)]
	\item 若$\int_{T}|x(t)|^2\dif t<\infty\Rightarrow a_{k}$有限,即平方可积收敛
				\[e(t) = \tilde{x}(t) = \sum_{k=-\infty}^{+\infty}a_{k}\e^{jk\omega_0t}\]
	\item 绝对可积:$\int_{T}|x(t)|\dif t\Rightarrow a_{k}$有限
\end{enumerate}
常用信号的Fourier级数:
\begin{enumerate}[label=(\arabic*)]
	\item $\tilde{x}(t) = \cos 100\uppi t = \frac{1}{2}\e^{j 100\uppi t}+\frac{1}{2}\e^{-j 100\uppi t}$
	\item $\tilde{x}(t) = \sin 100\uppi t = \frac{1}{2j}\e^{j 100\uppi t}+\frac{1}{2j}\e^{-j 100\uppi t}$
	\item $\delta_{T}(t) = \sum_{n=-\infty}^{+\infty}\delta(t-nT)\Rightarrow a_{k} = \frac{1}{T}\int_{-\frac{T}{2}}^{\frac{T}{2}}\delta(t)\e^{-jk\omega_0 t}\dif t = \frac{1}{T}$\\
				$\delta_T(t) = \frac{1}{T}\sum_{k=1}^{+\infty}\frac{2}{T}\cos(k\omega_0t) = \frac{1}{T}\sum_{k=-\infty}^{+\infty}\e^{jk\omega_0 t}$
	\item 周期脉冲串:
				\begin{figure}[H]
						\centering
						\begin{tikzpicture}[>=Stealth]
							\draw [->](-2,0)--(5,0)node[below]{$t$};
							\draw [->](0,-1)--(0,2)node[right]{$\tilde{x}(t)$};
							\draw (-1,0)node[below]{$-T_1$}--(-1,1)--(0,1)node[above left]{1}--(1,1)--(1,0)node[below]{$T_1$};
							\draw (-1.8,0)node[below]{$-\frac{T_0}{2}$} (1.8,0)node[below]{$\frac{T_0}{2}$} (3.6,0)node[below]{$T_0$};
							\draw (2.6,0)--(2.6,1)--(4.6,1)--(4.6,0);
							\filldraw (-1.6,0)circle(1pt) (-1,0)circle(1pt) (0,0)circle(1pt) (1,0)circle(1pt) (1.6,0)circle(1pt) (2.6,0)circle(1pt)	(3.6,0)circle(1pt) (4.6,0)circle(1pt);
				\end{tikzpicture}
				\end{figure}
				\[\begin{aligned}
					a_{k} &= \frac{1}{T_0}\int_{-T_1}^{T_1}\e^{-jk\omega_0 t}\dif t\\
					&= \frac{1}{T_0}\frac{1}{-jk\omega_0 t}\e^{-jk\omega_0 t}\bigg|_{-T_1}^{T_1} = \frac{1}{T_0}\frac{1}{-jk\omega_0 T_1}\e^{-jk\omega_0 T_1}-\frac{1}{T_0}\frac{1}{-jk\omega_0 T_1}\e^{jk\omega_0 T_1}\\
					&= \frac{2}{T_0}\frac{\sin k\omega T_1}{k\omega T_1} = \frac{2T_1}{T_0}\Sa{k\omega_0 T_1}\qquad (\Sa{x} = \frac{\sin x}{x})
				\end{aligned}\]
\end{enumerate}

DT(Discrete Time)周期信号$\tilde{x}[n]$的Fourier级数(DFS):\\
$\tilde{x}[n],N,\omega_0=\frac{2\uppi}{N},\{\e^{jk\omega_0 n}\}$\\
$\tilde{x}[n] = \sum_{k=0}^{N-1}a_{k}\e^{jk\omega_0 n},\qquad a_{k}=a_{k+N} = \frac{1}{N}\sum_{n=\mangle{N}}\tilde{x}[n]\e^{-jk\omega_0 t},\quad a_{k}=|a_{k}|\e^{j\theta_k}$\\
常用Fourier级数:\begin{enumerate}[label=(\arabic*)]
	\item $\tilde{x}[n] = \cos \frac{3\uppi}{5}n =\cos \frac{3}{10}2\uppi n = \frac{1}{2}\e^{j3\frac{2\uppi}{10}n} + \frac{1}{2}\e^{-j3 \frac{2\uppi}{10}n}$
	\item $\tilde{x}[n] = \sum_{k=-\infty}^{+\infty}\delta(n-kN) = \frac{1}{N}\sum_{k=0}^{N-1}\e^{jk\omega_0 n}\quad a_{k}=\frac{1}{N}$
\end{enumerate}

\subsection{傅里叶变换与傅里叶反变换}
对连续时间的信号$x(t)$的傅里叶变换:
\begin{enumerate}[label=(\arabic*)]
	\item 先说傅里叶级数和傅里叶变换的关系:傅里叶级数时对周期信号$\tilde{x}(t)$,对于非周期信号$x(t)$则写不出Fourier级数,所以使用傅里叶变换。详见\href{https://www.zhihu.com/question/21665935}{知乎}
	\item 傅里叶变换:对$x(t)$,其傅里叶变换(CTFT)为
				\begin{equation}
					X(\omega) = X(j\omega) = \int_{-\infty}^{+\infty}x(t)\e^{-j\omega t}\dif t
				\end{equation}
				或者\begin{equation}
					X(jf) = \int_{-\infty}^{+\infty}x(t)\e^{-j_2\uppi ft}\dif t\qquad f=\frac{\omega}{2\uppi}
				\end{equation}
				你可能要问$X(\omega)$和$X(j\omega)$怎么就相等呢?实际上就是两种表示罢了,参见\href{https://math.stackexchange.com/questions/966414/difference-between-x-omega-and-xj-omega-notation-of-fourier-transform}{MSE}
	\item 傅里叶反变换(还有个像上面一样用频率$f$的等价式子,略去不写):
				\begin{equation}
					f(t) = \frac{1}{2\uppi}\int_{-\infty}^{+\infty} X(j\omega)\e^{j\omega t} \dif\omega
				\end{equation}
\end{enumerate}
对离散时间的信号$x[n]$的傅里叶变换:
\begin{enumerate}[label=(\arabic*)]
	\item 先说说离散傅里叶变换(DFT)和离散时间傅里叶变换(DTFT)的区别:DTFT是对原信号在时域离散,DFT是对DTFT在频域上离散,相当于对原信号在时域、频域上都离散。详见\href{https://www.zhihu.com/question/23137926}{知乎}
	\item 接下来讲的都是DTFT:
\end{enumerate}

\appendix
\section{怪话表}
\begin{table}[H]
	\centering
	\begin{tabular}{ccc}
		\toprule
		中文 & 符号/缩写 & 说明 \\
		\midrule
		输入信号 & $x(t)$ & $e(t)$同义 \\
		输出信号 & $y(t)$ & $r(t)$同义 \\
		关系式 & $x(t)\longrightarrow y(t)$ & $r(t)=T\{e(t)\}$ \\
		单位冲激函数 & $\delta(t)$ &  \\
		单位阶跃函数 & $u(t)$ & 也有记号$\theta(t),\varepsilon(t)$ \\
		信号的偶分量 & $x_{e}(t)=\Ev{x(t)}$ & $\Ev{x(t)}=\frac{x(t)+x(-t)}{2}$,Even \\
		信号的奇分量 & $x_{o}(t)=\Od{x(t)}$ & $\Od{x(t)}=\frac{x(t)-x(-t)}{2}$,Odd \\
		$x(t)/x[n]$信号的能量 & $E_x$ & $E_x=\int_{-\infty}^{+\infty}|x(t)|^2\dif t$或$E_x = \sum_{n=-\infty}^{+\infty}|x[n]|^2$ \\
		$x(t)/x[n]$信号的功率 & $P_x$ & $P_x = \lim_{T \to \infty}\frac{1}{2T}\int_{-\infty}^{+\infty}|x(t)|^2\dif t$或$\lim_{N \to \infty}\frac{1}{2N+1}\sum|x[n]|^2$ \\
		线性时不变(非时变) & LTI & Linear Time Invariant \\
		齐次解/自由(固有)响应 & $y_{H}(t)$ & Homogeneous \\
		特解/强迫响应 & $y_{P}(t)$ & Particular \\
		零输入响应 & $y_{\zi}(t)$ & Zero Input \\
		零状态响应 & $y_{\zs}(t)$ & Zero State \\
		离散时间 & CT & Continuous Time \\
		连续时间 & DT & Discrete Time \\
		离散时间傅里叶级数 & DFS & Discrete Fourier Series \\
		连续时间傅里叶级数 & CFS & Continuous Fourier Series \\
		离散傅里叶变换 & DFT & Discrete Fourier Transform \\
		离散时间傅里叶变换 & DTFT & Discrete Time Fourier Transform \\
		连续时间傅里叶变换 & CTFT & Continuous Time Fourier Transform \\
		\bottomrule
	\end{tabular}
\end{table}

\section{草稿区}

\end{document}
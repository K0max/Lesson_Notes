\usepackage{NotesTeX}%笔记宏包
\usepackage{ctex}%提供中文
%------------------Math Package-------------------
\usepackage{amsmath}%提供数学等
\usepackage{amssymb}%一些特殊但是也常用的数学符号
\usepackage{amsthm}%数学环境设置
\usepackage{upgreek}%直立体pi等
\usepackage{esvect}%提供向量箭头,命令为\vv{*}
\usepackage{esint}%提供\oiint
\usepackage{extarrows}%提供箭头和等号
\usepackage{bm}%提供数学环境下的粗体
\usepackage{xcolor}
%------------------utility-------------------
%\usepackage[most]{tcolorbox}%提供盒子环境,用来制作theorem等环境
\usepackage{booktabs}%三线表
\usepackage{enumitem}%提供enumerate等环境label的设置
\usepackage{siunitx}%国际制单位
\usepackage{framed}%实现方框效果
%------------------Graphics-------------------
\usepackage{tikz}%TikZ绘图包
\usetikzlibrary{arrows.meta}%提供不同的TikZ箭头
\usetikzlibrary{decorations.pathreplacing}%提供不同的TikZ大括号
%\usepackage{newtxtext}
%------------------Page Design-------------------
\usepackage{geometry}%提供页面设计
\geometry{a4paper,centering,scale=0.8}

\usepackage{caption}%可以在minipage中使用\caption命令
%\usepackage{lipsum} % 该宏包是通过 \lipsum 命令生成一段本文,即乱序假文,正式使用时不需要引用该宏包

\usepackage{hyperref}%提供超链接

% ------------------Math Theorem environment-------------------
%\newtcbtheorem[number within = section]{definition}{定义}{%
%    boxrule=0pt,
%    boxsep=0pt,
%    colback={white!90},
%    enhanced jigsaw,
%    borderline west={2pt}{0pt}{red},
%    sharp corners,
%    before skip=10pt,
%    after skip=10pt,
%    breakable,
%}{mydefinition}

\newcommand{\const}{\mathrm{const}}
\newcommand{\defeq}{\xlongequal{\mathrm{def}}}%定义符号,也可以直接用amssymb的\triangleq

\renewcommand\i{\mathrm{i}}%i的定义

\newcommand*{\dif}{\mathop{}\!\mathrm{d}}
\DeclareMathOperator{\e}{e}
\DeclareMathOperator{\bit}{bit}
\DeclareMathOperator{\bits}{bits}
\renewcommand\vec[1]{\vv{\bm{#1}}}
\newcommand\ceil[1]{\left\lceil#1\right\rceil}
\newcommand\floor[1]{\left\lfloor#1\right\rfloor}
\newcommand\mangle[1]{\langle#1\rangle}
%\newcommand\Re[1]{\mathrm{Re}\left(#1\right)}
%\newcommand\Im[1]{\mathrm{Im}\left(#1\right)}
\renewcommand{\oiint}{\varoiint}
\newcommand{\red}{\color{red}}
\newcommand{\blue}{\color{blue}}
\newcommand{\green}{\color{green}}

\renewcommand{\geq}{\geqslant}
\renewcommand{\leq}{\leqslant}
\newcommand*{\E}{\mathbb{E}}
\renewcommand*{\P}{\mathbb{P}}
\newcommand*{\cov}{\mathrm{Cov}}
\renewcommand*{\var}{\mathrm{Var}}
\newcommand*{\iid}{\mathop{}\!\mathrm{i.i.d.}}
\newcommand{\injective}{\hookrightarrow}
\newcommand{\surjective}{\twoheadrightarrow}
\newcommand{\bijective}{\stackrel{\sim}{\rightarrow}}

\newcommand\belownote[2]{\mathop{#1}\limits_{\substack{#2}}}
\newcommand\abovenote[2]{\mathop{#1}\limits^{\substack{#2}}}
\newcommand\dbelownote[2]{\mathop{#1}\limits_{\substack{\downarrow \\ \text{#2}}}}
\newcommand\ubelownote[2]{\mathop{#1}\limits_{\substack{\uparrow \\ \text{#2}}}}
\newcommand\dabovenote[2]{\mathop{#1}\limits^{\substack{\downarrow \\ \text{#2}}}}
\newcommand\uabovenote[2]{\mathop{#1}\limits^{\substack{\uparrow \\ \text{#2}}}}

% 行内公式可编号,命令为\inlineequation[标签]{公式}
\makeatletter
\newcommand*{\inlineequation}[2][]{%
  \begingroup
    % Put \refstepcounter at the beginning, because
    % package `hyperref' sets the anchor here.
    \refstepcounter{equation}%
    \ifx\\#1\\%
    \else
      \label{#1}%
    \fi
    % prevent line breaks inside equation
    \relpenalty=10000 %
    \binoppenalty=10000 %
    \ensuremath{%
      % \displaystyle % larger fractions, ...
      #2%
    }%
    ~\@eqnnum
  \endgroup
}
\makeatother
\documentclass{article}

\usepackage{NotesTeX}%笔记宏包
\usepackage{ctex}%提供中文
%------------------Math Package-------------------
\usepackage{amsmath}%提供数学等
\usepackage{amssymb}%一些特殊但是也常用的数学符号
\usepackage{amsthm}%数学环境设置
\usepackage{upgreek}%直立体pi等
\usepackage{esvect}%提供向量箭头,命令为\vv{*}
\usepackage{esint}%提供\oiint
\usepackage{extarrows}%提供箭头和等号
\usepackage{bm}%提供数学环境下的粗体
\usepackage{xcolor}
%------------------utility-------------------
%\usepackage[most]{tcolorbox}%提供盒子环境,用来制作theorem等环境
\usepackage{booktabs}%三线表
\usepackage{enumitem}%提供enumerate等环境label的设置
\usepackage{siunitx}%国际制单位
\usepackage{framed}%实现方框效果
%------------------Graphics-------------------
\usepackage{tikz}%TikZ绘图包
\usetikzlibrary{arrows.meta}%提供不同的TikZ箭头
\usetikzlibrary{decorations.pathreplacing}%提供不同的TikZ大括号
%\usepackage{newtxtext}
%------------------Page Design-------------------
\usepackage{geometry}%提供页面设计
\geometry{a4paper,centering,scale=0.8}

\usepackage{lipsum} % 该宏包是通过 \lipsum 命令生成一段本文,即乱序假文,正式使用时不需要引用该宏包

\usepackage{hyperref}%提供超链接

% ------------------Math Theorem environment-------------------
%\newtcbtheorem[number within = section]{definition}{定义}{%
%    boxrule=0pt,
%    boxsep=0pt,
%    colback={white!90},
%    enhanced jigsaw,
%    borderline west={2pt}{0pt}{red},
%    sharp corners,
%    before skip=10pt,
%    after skip=10pt,
%    breakable,
%}{mydefinition}

\newcommand{\const}{\mathrm{const}}
\newcommand{\defeq}{\xlongequal{\mathrm{def}}}%定义符号,也可以直接用amssymb的\triangleq

\newcommand*{\dif}{\mathop{}\!\mathrm{d}}
\DeclareMathOperator{\e}{e}
\renewcommand\vec[1]{\vv{\bm{#1}}}
\newcommand\ceil[1]{\left\lceil#1\right\rceil}
\newcommand\floor[1]{\left\lfloor#1\right\rfloor}
\newcommand\mangle[1]{\langle#1\rangle}
\renewcommand{\oiint}{\varoiint}
\newcommand{\red}{\color{red}}
\newcommand{\blue}{\color{blue}}
\newcommand{\green}{\color{green}}

\renewcommand{\geq}{\geqslant}
\renewcommand{\leq}{\leqslant}
\newcommand*{\E}{\mathbb{E}}
\renewcommand*{\P}{\mathbb{P}}
\newcommand*{\cov}{\mathrm{Cov}}
\renewcommand*{\var}{\mathrm{Var}}
\newcommand*{\iid}{\mathop{}\!\mathrm{i.i.d.}}
\newcommand{\injective}{\hookrightarrow}
\newcommand{\surjective}{\twoheadrightarrow}
\newcommand{\bijective}{\stackrel{\sim}{\rightarrow}}
\usepackage{tkz-euclide}

\renewcommand\var[1]{\Delta#1}
\everymath{\displaystyle}
\newenvironment{solution}[1][]{\\\noindent\textcolor{blue}{Solution} \textcolor{red}{#1}\\ }%这里的参数指的是解法1或解法2
%\setlength{\parindent}{0pt}

\title{Quantum Physics}
\author{ypa}
\date{2023.3.6\\
Last Modified:\today
}

\begin{document}
\maketitle
\tableofcontents

\setcounter{section}{-1}
\section{几何光学}
\begin{enumerate}[label=\arabic*]
	\item $\frac{\sin\theta_1}{v_1}=\frac{\sin \theta_2}{v_2},n_1\sin \theta_1=n_2\sin \theta_2$
	\item 费马原理:光线从空间一点传播到另一点,沿所需{\red 时间为驻值}的路径传播\\
				$\displaystyle \delta t=0,t=\int_{A}^{B}\frac{\dif s}{v},l=ct=\int_{A}^{B} \frac{c}{v}\dif s=\int_{A}^{B}n\dif s\qquad \delta l=0$\\
				$n=n(x,y,z)$,光线经过的路径:$x=x(\lambda),\cdots $,光程为:
				\[l[x,y,z]=\int_{A}^{B}n\dif s=\int_{A}^{B}n(x,y,z)\sqrt{\dot{x}^2+\dot{y}^2+\dot{z}^2}\dif \lambda\]
				由变分法得程函方程:\[\frac{\partial n}{\partial x}-\frac{\dif }{\dif x}(nx')=0,\cdots \qquad \frac{\dif }{\dif s}\left(n\frac{\dif \vec{r}}{\dif s}=\nabla n\right)\]
\end{enumerate}

\section{量子物理的现象}
\subsection{经典物理的回顾}
\noindent$p=\frac{m_0v}{\sqrt{1-(\frac{v}{c})^2}}\quad E=\frac{m_0c^2}{\sqrt{1-(\frac{v}{c})^2}}\quad E=\sqrt{c^2p^2+m_0^2c^{\red 4}}$\\
给出$\red E=mc^2\quad m=\sqrt{m_0^2+\frac{p^2}{c^2}}$($m_0$为静质量)\\
$E_{k}=E-E_0=mc^2-m_0c^2$(本式是通式,在经典力学即低速运动中,由近似可以得到$E_{k}=\frac{1}{2}m_0v^2$)
\begin{example}
	质子+中子$\to$氘核的反应,该聚变反应释放的能量
	\begin{solution}
		质子质量$m_{p}$,中子质量$m_{n}$,氘核质量$m_{D}$,$E=\var{m}c^2=(m_{p}+m_{n}-m_{D})c^2$
	\end{solution}
\end{example}
\noindent 波动的概念:$E=E_0\cos (\omega t+\varphi_0)$\\
波的干涉:$E_1=E_{01}\cos (\omega t+\varphi_{01})\quad E_2=E_{02}\cos (\omega t+\varphi_{02})$\\
合成波:$E=E_1+E_2$\\
合成波强度:
\[
	\begin{aligned}
		I &= E^2=(E_1+E_2)^2=E_{01}^2+E_{02}^2+E_{01}E_{02}\cos\varphi_{12}\\
		\varphi_{12}\mn{当$\varphi_{12}=n\uppi$时$I$最大:干涉相长\\ 当$\varphi_{12}=(n+\frac{1}{2})\uppi$时最小:干涉相消} &= \varphi_1-\varphi_2 = \omega t_1+\varphi_{01}-\omega t_2-\varphi_{02} = \omega(t_1-t_2)\varphi_{01}-\varphi_{02}\\
		&= \omega(\frac{r_1}{v}-\frac{r_2}{v})+\varphi_{01}-\varphi_{02}\\
		&= \frac{2\uppi}{\lambda}(r_1-r_2)+(\varphi_{01}-\varphi_{02})\\
		\var{\varphi} &= \frac{2\uppi}{\lambda}\underbrace{\var{L}}_{\text{路程差}}+\underbrace{\var{\varphi_0}}_{\text{初始相位差}}
	\end{aligned}
\]
省流:两个波的合成或干涉依赖于两者的波程差与波长\\
\begin{tabular}{l|cc|l|cc}
	\hline
	 		 & 低速 & 牛顿力学 &    & 电磁学   & 波动 \\
	粒子 & 高速 & 相对论   & 波 & 量子物理 & 波粒二象性 \\
			 & 能量、动量     &    &          & 波的干涉 \\ 
	\hline
\end{tabular}

\subsection{电磁波的量子化}
\begin{enumerate}[label=(\arabic*)]
	\item 黑体辐射:\\
				实验内容:热的物体发射电磁波的现象\\
				实验结果:电磁波强度与频率的关系
				$E=h\nu$:电偶极子的能量是不连续的.$h=6.63\times 10^{-34}\si{J\cdot s}$
	\item 光电效应:\\
				实验内容:一个轻元素物质表面吸收电磁波的过程\\
				实验结果:光电流信号与电磁波强度无关,只与其频率有关\\
				有截止频率,电子数目与频率成正比:$E_e=\underbrace{h\nu }_{E_p:\text{电磁波的能量}}-W_0(\text{束缚能})$\\
				截止频率:$\nu_0=\frac{W}{h}$
	\item Compton散射:\\
				实验内容:X射线被电子散射的过程\\
				实验结果:X射线被电子散射,随着散射角度增长,X射线波长也发生改变。散射角$\theta \propto \var\lambda,\quad \lambda-\lambda'=\frac{2h}{mc}\sin ^2 \frac{\theta}{2}$
				\begin{figure}[H]
					\centering
					\begin{tikzpicture}[>=Stealth]
						\draw [->](0,0)node[left]{X光子}--(1,0);
						\draw [dashed](3,0)node[left]{电子}--(5.5,0);
						\draw [dashed](3,0)--(5,1)(3,0)--(4.5,-1);
						\filldraw [red](0,0)circle(1.5pt);
						\filldraw (3,0)circle(1pt);
						\filldraw [red](5,1)circle(1.5pt);
						\filldraw (4.5,-1)circle(1pt);
						\draw [red](3.8,0)arc(0:28:0.8);
						\draw (3.5,0)arc(0:-31:0.5);
						\draw (3.8,0.3)node[right]{$\red\theta$};
						\draw (3.5,-0.2)node[right]{$\theta'$};
					\end{tikzpicture}
				\end{figure}
				\begin{tabular}{cccc|ccc}
					初始 & X光子 & 能量 & $E=h\nu$ & 电子 & 能量 & $E=m_0c^2$ \\
							&       & 动量 & $p=\frac{h\nu}{c}$ & & 动量 & $0$ \\
					散射后 & X光子 & 能量 & $E=h\nu'$ & 电子 & 能量 & $E=mc^2=\frac{m_0c^2}{\sqrt{1-(\frac{v_e}{c})^2}}$ \\
								&       & 动量 & $p=\frac{h\nu'}{c}$ & & 动量 & $p=mv_e=\frac{m_0v_e}{\sqrt{1-(\frac{v_e}{c})^2}}$ \\
				\end{tabular}\\
				由能量守恒和动量守恒(水平方向和垂直方向)可得到三个方程式
				\begin{align}
					&\text{能量守恒} & h\nu+m_0c^2 &= h\nu'+\frac{m_0c^2}{\sqrt{1-(\frac{v_e}{c})^2}} \label{能量守恒eq1} \\
					&\text{动量守恒} & (\text{水平}) \quad \frac{h\nu}{c} &= \frac{h\nu'}{c}\cos\theta + \frac{m_0v_e}{\sqrt{1-(\frac{v_e}{c})^2}}\cos\theta' \label{水平动量eq2} \\
					&   & (\text{垂直}) \qquad 0 &= \frac{h\nu'}{c}\sin\theta - \frac{m_0v_e}{\sqrt{1-(\frac{v_e}{c})^2}}\sin\theta' \label{垂直动量eq3}
				\end{align}
				联立解得:
				\[
					\begin{aligned}
						\nu-\nu' &= \frac{2h}{m_0c^2}\nu \nu'\sin^2\frac{\theta}{2} \qquad \text{两边同除以$\nu,\nu'$,乘以$c$得到下式}\\
						\lambda'-\lambda &= \frac{2h}{m_0c}\sin^2\frac{\theta}{2}
					\end{aligned}
				\]
	\item 相对论:$E=\sqrt{c^2p^2+m_0^2c^{\red 4}}=mc^2$,$E_k=E-E_0=mc^2-m_0c^2$
				\begin{enumerate}[label=(\alph*)]
					\item 光:$m_0=0$,$E=cp=h\nu ^2$
					\item 电子:$m_0=m_e,m=\frac{m_0}{\sqrt{1-(\frac{v_e}{c})^2}}=\frac{m_e}{\sqrt{1-(\frac{v_e}{c})^2}}$,$p=mv_e=\frac{m_e v_e}{\sqrt{1-(\frac{v_e}{c})^2}}$,$E=mc^2=\frac{m_e c^2}{\sqrt{1-(\frac{v_e}{c})^2}}$,$m=\sqrt{m_0^2+\frac{p^2}{c^2}}$
				\end{enumerate}
				\begin{remark}
					上述$E_0$代表粒子的静能量,$E$代表其动能量,$E_k$代表其动能,动能定义为动能量和静能量的差。
					上述$m_0$代表粒子的静质量,$m$代表其动质量,$m_e$代表电子的静质量。
				\end{remark}
				\begin{example}
					一个电灯泡功率100W,设其为白光(平均波长500nm),问每秒发射多少光子?\\
					\textbf{Solution:\\}
					单个光子能量$E = h\nu = h\frac{c}{\lambda}$,总能量:$E_t=100\si{J}$,光子数$n=\frac{E_t}{E}=\frac{E_t\lambda}{hc}=2.5\times 10^{20}$
				\end{example}
\end{enumerate}

\subsection{量子结构的量子化}
\begin{enumerate}[label=(\arabic*)]
	\item 卢瑟福$\alpha$粒子($\text{He}_4^2$)散射实验:\\
				实验内容:$\alpha$粒子向金片散射\\
				实验结果:$\alpha$粒子向原子的散射过程中,$\alpha$粒子绝大部分直线穿过,只有极少数偏射甚至反射\\
				理论解释:汤姆逊模型(枣糕模型):电子和正电荷混杂在原子中(不能解释)\\
				卢瑟福模型:中心有一个原子核,电子围绕原子核
	\item	氢原子光谱:有一系列分立的光谱线,满足巴耳末公式$\nu=R_H c(\frac{1}{n_1^2}-\frac{1}{n_2^2})$,里德伯常数$R_H=1.096\times 10^{-7}\si{m^{-1}}$
	\item 玻尔原子模型(半经典半量子):量子化角动量假设:电子在轨道上做圆周运动,角动量(动量,半径)是量子化的:$L=mvr=n\hbar\quad \hbar=\frac{h}{2\uppi}$\\
				跃迁假设:电子从一个轨道跃迁到另一轨道,能量以电磁波形式发射,形成光谱线\\
				定量计算:\[\begin{cases}
					(\text{库仑力})\frac{1}{4\uppi \varepsilon_0}\cdot \frac{e^2}{r^2} = m \frac{v^2}{r}(\text{向心力})\\
					mv_e r = n\hbar
				\end{cases}\Rightarrow r_n = \frac{4e^2 \uppi^2 m^2r^2}{n^2h^2m_4\uppi\varepsilon_0} = \frac{e^2\uppi mr^2}{\varepsilon_0n^2h^2} = \frac{\varepsilon_0 n^2h^2}{e^2\uppi m}\propto n\text{故不连续}\]
				\textbf{注:}上述是氢原子,核内质子带电为$e$,核外电子也是$e$,若其他原子质子数为$k$,那么式子变成:
				\[\begin{cases}
					(\text{库仑力})\frac{1}{4\uppi \varepsilon_0}\cdot \frac{{\red k}e^2}{r^2} = m \frac{v^2}{r}(\text{向心力})\\
					mv_e r = n\hbar
				\end{cases}\Rightarrow r_n = \frac{4e^2 \uppi^2 m^2r^2}{n^2h^2m_4\uppi\varepsilon_0} = \frac{e^2\uppi mr^2}{\varepsilon_0n^2h^2} = \frac{\varepsilon_0 n^2h^2}{{\red k}e^2\uppi m}\]
				电子绕核转的能量:
				\[\begin{split}
					E &= E_{\text{动}}+E_{\text{势}} = \frac{1}{2}mv^2-\frac{e^2}{r}\cdot \frac{1}{4\uppi \varepsilon_0}\\
					&= \frac{n^2h^2}{8\uppi^2 mr^2}-\frac{e^2}{4\uppi\varepsilon_0}\cdot \frac{e^2 \uppi m}{\varepsilon_0n^2h^2}\\
					&= -\frac{e^4m}{8\varepsilon_0^2n^2h^2}-\frac{e^4m}{4\varepsilon_0^2n^2h^2}\\
					&= -\frac{me^4}{8\varepsilon_0^2n^2h^2}\sim \frac{1}{n^2}
				\end{split}\]
				从$n_1$跃迁到$n_2$:$\var{E} = E_{n_1}-E_{n_2} = \frac{me^4}{8\varepsilon_0^2h^2}\left(\frac{1}{n_1^2}-\frac{1}{n_2^2}\right)$
				$\var{E}$会以电磁波的形式放射出去:$\var{E}=h\nu \Rightarrow \nu = \frac{E}{h}=\underbrace{\frac{me^4}{8\varepsilon_0^2h^3}}_{R_H\cdot c}\left(\frac{1}{n_1^2}-\frac{1}{n_2^2}\right)$,里德伯常量$R_H=\frac{me^4}{8\varepsilon_0^2h^3c}$
	\item Frank-Hertz实验:\\
				实验内容:原子吸收电子的过程\\
				实验结果:形成一系列的峰谷结构,且每个峰/谷的间距都是$4.9\si{V}$\\
				理论解释:Hg原子的能级是量子化的
\end{enumerate}

\subsection{波粒二象性}
\begin{enumerate}[label=(\arabic*)]
	\item 描述波:波长$\lambda$,周期$T$,频率$\nu$\quad 描述粒子:动量$p$,能量$E$\\
				如何将左右的物理量联系起来?\\
				普朗克公式$E=h\nu$,爱因斯坦公式:$E=cp$,于是$cp=h\nu \Rightarrow p=\frac{h\nu}{c}$,又由$\lambda=\frac{c}{\nu}$,有$p=\frac{h}{\lambda}$\\
				德布罗意波粒二象性公式:$E=h\nu,p=\frac{h}{\lambda}$.光,电子,以及一切微观粒子都有波粒二象性\\
				\textbf{推论:}\begin{enumerate}[label=(\alph*)]
					\item 一般情况:$\lambda=\frac{h}{p}$,$\lambda=\frac{h}{\sqrt{(\frac{E_k}{c})^2+2m_0E_k}}$\\
								推导:$E=\sqrt{c^2p^2+m_0^2c^4},E_k=E-m_0c^2\Rightarrow p=\sqrt{\left(\frac{E_k}{c}\right)^2+2m_0E_k}$
					\item 非相对论:$m\approx m_0\Rightarrow E_k=\frac{p^2}{2m}\Rightarrow p=\sqrt{2mE_k}\quad \lambda=\frac{h}{p}=\frac{h}{\sqrt{2mE_k}}$\\
								有温度情况:$E=\uppi k_B T$(温度是物体热运动的表现),得$\lambda=\frac{h}{\sqrt{2mE_k}}=\frac{h}{\sqrt{2m\uppi k_B T}}$
				\end{enumerate}
	\item 光的波粒二象性:粒子性:黑体辐射、光电效应;波动性:干涉衍射、光学显微镜、光谱\\
				光波动性——晶格衍射\footnote{下面明明讲的是干涉,为啥这一行说是衍射呢?因为他俩一般没啥区别,详见知乎\href{https://www.zhihu.com/question/30233166}{干涉和衍射的区别}},光的干涉:\\
				光的干涉现象:两个电磁波相干叠加的效应.出现明暗相间的条纹\\
				理论解释:电磁波的波动性.
				\[\underbrace{E}_{\text{电磁波}} = \underbrace{A}_{\text{振幅}}\cos (\omega t-kr+\phi_0)\]
				圆频率$\omega=\frac{2\uppi}{T}$(时间上的周期性),波矢$k=\frac{2\uppi}{\lambda}$(空间上的周期性)\\
				\begin{itemize}
					\item (1号波)$E_1 = A_1\cos (\omega t-kr_1+\phi_0)$
					\item (2号波)$E_2 = A_2\cos (\omega t-kr_2+\phi_0)$
				\end{itemize}
				在$P$点两个波叠加:$E = E_1+E_2 = A_1\cos (\omega t-kr_1+\phi_0)+A_2\cos (\omega t-kr_2+\phi_0)$\\
				场强:$I = A_1^2+A_2^2+2A_1A_2\cos (\var \phi)$,$\var \phi=\phi-\phi_2 = -k(r_1-r_2)$
				则有\[I = A_1^2+A_2^2+2A_1A_2\cos (\underbrace{k\var L}_{\text{光程差}})\]
				当$k\var L = 2n\uppi$时,$I_{\max} = A_1^2+A_2^2+2A_1A_2$,当$k\var L=(2n+1)\uppi$时,$I_{\min} = A_1^2+A_2^2-2A_1A_2$\\
				由$\max:k=\frac{2\uppi}{\lambda}$,有$\frac{2\uppi}{\lambda}\var L = 2n\uppi \Rightarrow \var L=n\lambda$,\quad $\min:\var L = \left(n+\frac{1}{2}\right)\lambda$
				\begin{minipage}{0.3\textwidth}
					\begin{tikzpicture}[decoration={brace, amplitude=5}]
						\draw (0,-1.5)--(0,-1.1)(0,-0.9)--(0,0.9)(0,1.1)--(0,2.5)(3,-1.5)--(3,2.5);
						\draw (0,1)node[left]{$s_1$};
						\draw (0,-1)node[left]{$s_2$};
						\draw [decorate](-0.05,-1)--(-0.05,1);
						\draw (-0.2,0)node[left]{$d$};
						\draw (3,2)node[right]{$P$};
						\filldraw (3,2)circle(1.5pt);
						\draw [red](0,-1)--(3,2);
						\draw [red](0,1)--(3,2);
						\draw [red,dashed](0,1)--(1,0);
						\draw (0,1)coordinate (s1) (0,-1)coordinate (s2) (1,0)coordinate (d);
						\tkzMarkRightAngle(s1,d,s2);
						\draw (0,0.7)arc(-90:-45:0.3);
						\draw (0.2,0.7)node[below]{$\theta$};
						\draw [decorate](1,0)--(0,-1);
						\draw (0.5,-0.5)node[below right]{$\var{L}$};
					\end{tikzpicture}
				\end{minipage}
				\hfill
				\begin{minipage}{0.6\textwidth}
					$\var{L} = d\sin\theta$\\
					$\var{L} = d\sin\theta = n\lambda \quad (n\in\mathbb{N}^*)$(干涉相长,亮纹)\\
					$\var{L} = d\sin\theta = \left(n+\frac{1}{2}\right)\lambda \quad (n\in\mathbb{N}^*)$(干涉相消,暗纹)
				\end{minipage}
	\item X射线的波粒二象性:粒子性:Compton散射;波动性:X射线晶格衍射、X射线显微镜、X射线光谱\\
				X射线晶格衍射:\\
				实验内容:X射线/光经过晶体发生干涉现象\\
				实验结果:明暗相间条纹\\
				理论解释:X射线也是一种波,波长为$\lambda$\\
				\begin{minipage}{0.25\textwidth}
					\begin{tikzpicture}[>=Stealth,decoration={brace, amplitude=5}]
						\foreach \i in {0,1,2}{
							\foreach \j in {0,1,2}
								\draw (\i,\j)circle(1.5pt);
						}
						\draw [->,red](-0.2,2.2)--(1,1)--(2.4,2.4);
						\draw [->,red](-0.2,1.1)--(1,0)--(2.4,1.4);
						\draw [decorate](1,1)--(1,0);
						\node at (1.35,0.5){$d$};
						\draw [dashed](1,1)--(0.5,0.5)(1,1)--(1,0);
						\draw (0.6,0.6)--(0.7,0.5)--(0.6,0.4);
						\draw [blue](0.75,0.75)arc(-135:-80:0.25);
						\node at (0.7,0.85){$\blue\theta$};
					\end{tikzpicture}
				\end{minipage}
				\hfill
				\begin{minipage}{0.8\textwidth}
					$\var{L}={\red 2}d\sin\theta=n\lambda \quad (n\in\mathbb{N}^*)$(干涉相长,亮纹)\\
					$\var{L}={\red 2}d\sin\theta=\left(n+\frac{1}{2}\right)\lambda \quad (n\in\mathbb{N}^*)$(干涉相消,暗纹)\\
					$\var{L}$是光程差,$d$是晶格距离,比如钠原子的距离\\
					{\red 注意:}和光的干涉相比,X射线晶格衍射的光程差多了上式标红的2
				\end{minipage}
	\item 电子的波粒二象性:粒子性:Compton散射;波动性:衍射(同上述X射线晶格衍射)、电子显微镜、电子光谱
	\item	中子的波粒二象性:粒子性:中子射线;波动性:中子显微镜、衍射
	\item	原子的波粒二象性:粒子性:Bohr氢原子理论、F-H实验;波动性:“巨原子”实验、超流BEC\\
				$\lambda=\frac{h}{\sqrt{2\uppi mk_BT}}$质量小,温度低,波长大,波动性明显
	\item 分子的波粒二象性:粒子性:化学反应基本粒子;波动性:衍射。$\text{C}_{60}$的衍射实验
\end{enumerate}

\section{量子物理的基本理论框架}
\subsection{单电子双缝干涉实验}
现象:\begin{enumerate}[label=(\arabic*)]
	\item 单个电子落在屏幕上,随机
	\item 多次重复,屏幕上出现明暗条纹
\end{enumerate}

\subsection{波函数}
微观世界单个粒子的运动状态,用波函数来描述$\Psi(\vec{r},t)$

\section{公式区}
\begin{enumerate}[label=(\alph*)]
	\item $E=h\nu,\quad p=\frac{E}{c}=\frac{h\nu}{c}=\frac{h}{\lambda}$
	\item $v=\lambda \nu,\quad p=\frac{h}{\lambda}$
\end{enumerate}

\section{草稿区}
\[\frac{\dif}{\dif x}\frac{\dif y}{\dif y'}=\frac{\dif}{\dif y'}\frac{\dif y}{\dif x}=\frac{\dif}{\dif y'}y'=1\]

\end{document}